\documentclass{article}
\usepackage[a4paper, total={6in, 9in}]{geometry}
\usepackage{fancyhdr}
\usepackage{amsmath}
\usepackage{algorithm}
\usepackage{algpseudocode}
\usepackage{CJK}
\usepackage{graphicx}

\newtheorem{theorem}{Theorem}
\newtheorem{lemma}{Lemma}
\newtheorem{proof}{Proof}[section]

\pagestyle{fancy}
\fancyhf{}
\rhead{1600017857}
\lhead{Homework(2)}
\rfoot{}
\cfoot{Page \thepage}

\title{Homework(2)}
\date{2018-03-02}
\author{
  \begin{CJK}{UTF8}{gbsn}
    黄道吉-1600017857
  \end{CJK}
}

\begin{document}
\begin{CJK}{UTF8}{gbsn}
\section{Q1}
  \paragraph{$\mathcal{C}_1 = \phi$}
    是开集, 闭集, 有界集, 紧集, 它的内点集, 闭包, 边界和聚点集都是$\phi$.
  \paragraph{$\mathcal{C}_2 = \mathcal{R}^n$}
    是开集, 闭集, 无界集, 不是紧集, 它的内点集, 闭包, 边界和聚点集都是$\mathcal{R}^n$ .
  \paragraph{$\mathcal{C}_3 = [0, 1) \cup [2, 3] \cup (4, 5]$}
    不是开集, 不是闭集, 有界, 不紧, 它的内点集是 $(0, 1) \cup (2, 3) \cup (4, 5)$, 它的闭包是 $[0, 1] \cup [2, 3] \cup [4, 5]$, 它的边界是 $\{0, 1, 2, 3, 4, 5\}$, 它的聚点集是 $[0, 1] \cup [2, 3] \cup [4, 5]$.
  \paragraph{$\mathcal{C}_4 = \{(x, y)^T |\  x \geq 0, y > 0\}$}
    不开, 不闭, 无界, 不紧, 它的内点集是 $\{(x, y)|\ x > 0, y > 0\}$, 它的闭包是 $\{(x, y)|\ x \ge 0, y \ge 0\}$, 它的边界是 $\{(0, y)|y \ge 0 \} \cup \{(x, 0)|\ x \ge 0\}$, 它的聚点集是 $\{(x, y)|\ x \ge 0, y \ge 0\}$.
  \paragraph{$\mathcal{C}_5 = \{k|\ k \in \mathcal{Z}\}$}
    不是开集, 是闭集, 无界, 不紧, 它的内点集是 $\phi$, 它的闭包和边界集都是它自身, $\{k|\ k \in \mathcal{Z}\}$, 它的聚点集是 $\phi$.
  \paragraph{$\mathcal{C}_6 = \{k^{-1}|\ k \in \mathcal{Z}\}$}
    不开, 不闭, 无界, 不紧, 它的内点集是 $\phi$, 它的闭包和边界集都是 $\{k^{-1}|\ k \in \mathcal{Z}\} \cup \{0\}$, 它的聚点集是 $\{0\}$.
  \paragraph{$\mathcal{C}_7 = \{(1 / k, \sin{k}^T|\ k \in \mathcal{Z})\}$}
    不开, 不闭, 有界, 但不紧, 它的内点集是 $\phi$, 它的闭包和边界集都是 $\{(1 / k, \sin{k}^T|\ k \in \mathcal{Z})\} \cup \{(0, y)|\ -1 \leq y \leq 1\}$, 它的聚点集是 $\{(0, y)|\ -1 \leq y \leq 1\}$.

\section{Q2}
  \paragraph{}
    1. 若 $\mathcal{C}$ 是闭集, 如果存在 $x^*$ 是 $\mathcal{C}$ 中一个收敛序列的极限点, 但 $x^* \not\in \mathcal{C}$, 所以有 $x^* \in \mathcal{C}^c$, 而 $\mathcal{C}^c$ 是一个开集, 所以有
    \begin{equation}
      \exists \epsilon\ s.t.\ (\cup(x^*, \epsilon)) \cap \mathcal{C} = \phi
    \end{equation}
    因为 $\cup(x^*, \epsilon) \subseteq \mathcal{C}^c$. 但存在 $\{x_k\}_1^{\infty} \subseteq \mathcal{C}$ 也即 $\lim_{k \to \infty} x_k = x^*$, 也就是说,
    \begin{equation}
      \forall \epsilon\ \ (\cup(x^*, \epsilon)) \cap \mathcal{C} \not= \phi
    \end{equation}
    矛盾! 所以对于所有 $x^*$ 是 $\mathcal{C}$ 中一个收敛序列的极限点, 都有 $x^* \in \mathcal{C}$.
    \paragraph{}
      2. 如果 $\mathcal{C}$ 包含其中所有收敛序列的极限点, 但 $\mathcal{C}$ 并不是闭集, $i.e.$ $\mathcal{C}^c$ 不是开集, 也就是,
      \begin{equation}
        \exists x^* \in \mathcal{C}^c\ \forall\ \epsilon > 0\ (\cup(x^*, \epsilon)) \cap \mathcal{C} \not= \phi
      \end{equation}
      因为 $(\mathcal{C}^c)^c = \mathcal{C}$. 选定一列 $\epsilon_k \to 0$ 并选取 $x_k \in (\cup(x^*, \epsilon)) \cap \mathcal{C}$, 有 $\lim_{k \to \infty} x_k = x^*$, 得到 $x^* \in \mathcal{C}$, 矛盾! 所以 $\mathcal{C}$ 是闭集.
    \paragraph{}
      3. 由1和2, 我们证明了 $\mathcal{C} \subseteq \mathcal{R}^n$ 是闭集 $\iff$ $\mathcal{C}$ 包含其中所有收敛序列的极限点.

\section{Q3}
  \paragraph{}
  $x$ $\in$ $\partial\mathcal{C} = \bar{\mathcal{C}} \setminus \mathcal{C}^o = ((\mathcal{C}^c)^o)^c \setminus \mathcal{C}^o = ((\mathcal{C}^c)^o)^c \cap (\mathcal{C}^o)^c$
  \paragraph{}
    由定义, $x \in \mathcal{C}^o \iff \exists \epsilon > 0\ \cup(x, \epsilon) \subseteq \mathcal{C}$, 所以
    \begin{equation}
      x \in (\mathcal{C}^o)^c\ \iff\ \forall \epsilon > 0\ \exists z \not\in \mathcal{C}\ |z - x|_2 < \epsilon
    \end{equation}
    把等式(4)中 $\mathcal{C}$ 替换成 $\mathcal{C}^c$, 得到
    \begin{equation}
      x \in ((\mathcal{C}^c)^o)^c\ \iff\ \forall \epsilon > 0\ \exists y \in \mathcal{C}\ |y - x|_2 < \epsilon
    \end{equation}
  \paragraph{}
    所以 $x$ $\in$ $\partial\mathcal{C} = ((\mathcal{C}^c)^o)^c \cap (\mathcal{C}^o)^c \iff \forall \epsilon > 0\ \exists y \in \mathcal{C}\ |y - x|_2 < \epsilon\ \exists z \not\in \mathcal{C}\ |z - x|_2 < \epsilon$

\section{Q4}
  \paragraph{1.1} 如果 $\mathcal{C}$ 是闭集,  那么 $\forall x^* \in \partial \mathcal{C}$, 由 $Q3$, 得到 $\forall \epsilon > 0\ \exists y \in \mathcal{C}\ |y - x^*|_2 < \epsilon$, 那么 $x^*$ 一定是 $\mathcal{C}$ 中一个收敛序列的极限点, 又由 $Q2$, 得到 $x^* \in \mathcal{C}$, 所以 $\mathcal{C} \supseteq \partial \mathcal{C}$.
  \paragraph{1.2} 如果 $\mathcal{C} \supseteq \partial \mathcal{C}$, 但 $\mathcal{C}$ 并不是闭集, 由 $Q2$ 的逆否命题, 至少存在一个点 $x^*$ 是 $\mathcal{C}$ 中一个收敛序列的极限点, 但 $x^* \not\in \mathcal{C}$, 所以有
  \begin{equation}
    \forall \epsilon > 0\ \exists y \in \mathcal{C}\ |y - x^*|_2 < \epsilon\ \exists z \not\in \mathcal{C}\ |z - x^*|_2 < \epsilon,
  \end{equation}
  只需选择 $z = x^*$, 那由 $Q3$, $x^* \in \partial \mathcal{C} \subseteq \mathcal{C}$, 矛盾! 所以 $\mathcal{C}$ 是一个闭集.
  \paragraph{2.1} 如果 $\mathcal{C}$ 是开集, 假设存在 $x^* \in \mathcal{C} \cap \partial \mathcal{C}$, 那么 $\exists \epsilon > 0\ \cup(x^*, \epsilon) \subseteq \mathcal{C}$ 并且 $\forall \epsilon > 0\ \exists z \not\in \mathcal{C}\ |z - x^*|_2 < \epsilon$, 矛盾! 所以 $\mathcal{C} \cap \partial \mathcal{C} = \phi$.
  \paragraph{2.2} 如果 $\mathcal{C} \cap \partial \mathcal{C} = \phi$, 假设 $\mathcal{C}$ 不是开集, $i.e.$ $\exists x^* \in \mathcal{C}\  \forall \epsilon > 0\ \exists z \not\in \mathcal{C}\ |z - x^*|_2 < \epsilon$, 那么由 $Q3$, $x^* \in \partial \mathcal{C}$
  , 矛盾! 所以 $\mathcal{C}$ 是开集.

\section{Q5}
  \paragraph{a.1}
    \begin{equation}
      \begin{aligned}
        \overline{\mathcal{A} \cup \mathcal{B}} & = (((\mathcal{A} \cup \mathcal{B})^c)^o)^c \\
        & = ((\mathcal{A}^c \cap \mathcal{B}^c)^o)^c \quad (De Morgan) \\
        & = ((\mathcal{A}^c)^o \cap (\mathcal{B}^c)^o)^c \quad (*) \\
        & = ((\mathcal{A}^c)^o)^c \cup ((\mathcal{B}^c)^o)^c \quad (De Morgan) \\
        & = \overline{\mathcal{A}} \cup \overline{\mathcal{B}}
      \end{aligned}
    \end{equation}
    下证(*)式, 只需证明 $(P \cap Q)^o = P^o \cap Q^o$
    \begin{equation}
      \begin{aligned}
        & \forall x \in \mathcal{R}^n \\
        & x \in (P \cap Q)^o \\
        \iff & \exists \epsilon \ \cup (x, \epsilon) \subseteq P \cap Q \\
        \iff & \exists \epsilon \ \cup (x, \epsilon) \subseteq P \ \land\  \exists \epsilon \quad \cup (x, \epsilon) \subseteq Q \\
        \iff & x \in P^o \ \land\  x \in Q^o \\
      \end{aligned}
    \end{equation}
    所以 $(P \cap Q)^o = P^o \cap Q^o$.
  \paragraph{a.2}
    $\forall x \in \overline{\mathcal{A} \cap \mathcal{B}}$, 存在 $\{x_k\}_1^{\infty} \subseteq \mathcal{A} \cap \mathcal{B}$, 并且 $\lim_{k \to \infty} x_k = x$ (只需证明$\overline{\mathcal{C}} = \mathcal{C} \cup \mathcal{C}^{'}$, 其中, $\mathcal{C}^{'}$是导集(聚点集), 由$Q2$, $\mathcal{C} \cup \mathcal{C}^{'} \subseteq \mathcal{C}$, 即一个集合的闭包必然包含其导集, 又$\mathcal{C} \cup \mathcal{C}^{'}$是闭集($\forall x \in (\mathcal{C} \cup \mathcal{C}^{'})^c$, 若$\forall \epsilon > 0$ 都有 $(\cup(x, \epsilon)) \cap (\mathcal{C} \cup \mathcal{C}^{'}) \not= \phi$ 则可以构造一列$x_k \in \mathcal{C}$, $\lim_{k \to \infty} = x$, 即 $x \in \mathcal{C}^{'}$, 矛盾!) 所以$\overline{\mathcal{C}} = \mathcal{C} \cup \mathcal{C}^{'}$, 因为闭包是最小闭集, 证毕)
    \paragraph{}
    所以 $\exists \{x_k\} \subseteq \mathcal{A},\ \lim_{k \to \infty} x_k = x$, $\exists \{y_k\} \subseteq \mathcal{B},\ \lim_{k \to \infty} y_k = x$,
    \paragraph{}
    所以 $x \in \mathcal{A}^{'} \subseteq \overline{\mathcal{A}}, x \in \mathcal{B}^{'} \subseteq \overline{\mathcal{B}}$, 即$x \in \overline{\mathcal{A}} \cap \overline{\mathcal{B}}$, 即 $\overline{\mathcal{A} \cap \mathcal{B}} \subseteq \overline{\mathcal{A}} \cap \overline{\mathcal{B}}$
    \paragraph{}
    例子: $\mathcal{R}$ 中, $\mathcal{A} = \{x |\ x < 0\},\ \mathcal{B} = \{x |\ x > 0\}$, 则 $\overline{\mathcal{A} \cap \mathcal{B}} = \phi$, 但 $\overline{\mathcal{A}} \cap \overline{\mathcal{B}} = \{0\}$.
\section{Q6}
  \paragraph{a}
    $\lim_{k \to \infty} x_k = 0$, 又 $\frac{|e_{k + 1}|}{|e_k|} = 0.5 < +\infty$, 所以$r = 1, c = 0.5$
  \paragraph{b}
    $\lim_{k \to \infty} x_k = 1$, 又 $\frac{|e_{k + 1}|}{|e_k|} = 0.01 < +\infty$, 所以$r = 1, c = 0.01$
  \paragraph{c}
    $\lim_{k \to \infty} x_k = 0$, 又 $\frac{|e_{k + 1}|}{|e_k|^2} = 1 < +\infty$, 所以$r = 2, c = 1$
  \paragraph{d}
    $\lim_{k \to \infty} x_k = 0$, 又 $\frac{|e_{k + 1}|}{|e_k|} = 3^{-2k - 1} \to 0 < +\infty$, 所以$r = 1, c = 0$
  \paragraph{e}
    $\forall k, |e_k| \leq 2^{-k}$, 由$Q6.a$得 $r = 1, c = 0.5$.

\section{Q7}
  可以确定
  \paragraph{}
  不妨设 $\lim_{k \to \infty} = x^*$, 类似的, 约定 $y^*, r_y, c_y, r_x, r_c, c_x, c_c$
  \begin{equation}
    \begin{aligned}
      \frac{|y_{k + 1} - y^*|}{|y_k - y^*|^{r_y}} & = \frac{|c_{k + 1}x_{k + 1} - c_{k + 1}x^* + c_{k + 1}x^* - cx^*|}{|c_{k}x_{k} - c_{k}x^* + c_{k}x^* - cx^*|^{r_y}} \\
      & = \frac{|c_{k + 1}e_{x, k + 1} + e_{c, k + 1}x^*|}{|c_{k}e_{x, k} + e_{c, k}x^*|^{r_y}} \\
      & = \frac{|ce_{x, k + 1} + e_{c, k + 1}x^*|}{|ce_{x, k} + e_{c, k}x^*|^{r_y}} \quad (c_k = c + e_{c, k} = c + o(c)) \\
    \end{aligned}
  \end{equation}
  \paragraph{}
    $x^* = 0$时, 上式等于$|e_{x, k + 1}| / |e_{x, k}|^{r_y}$, 则有 $r_y = r_x, c_y = c_x$.
  \paragraph{}
    $x^* \not= 0$时, 分析各项的阶, $r$越大则对应变量的阶越小, 则成为可以忽略的无穷小量, 所以
    \begin{equation}
      r_y = min\{r_x, r_c\},\quad c_y =
      \begin{cases}
        c_x,\ r_x < r_c \\
        c_c,\ r_c < r_x \\
      \end{cases}
    \end{equation}
  \paragraph{}
  而当$r_c = r_x$时, $c$越大, 它的阶越大($e_{k + 1} \approx c|e_{k}|^r$)
  \begin{equation}
    c_y = max\{c_x, c_c\}\ (when\ r_x = r_c)
  \end{equation}
  \paragraph{}
  综上所述
  $x^* = 0$时
  \begin{equation}
    r_y = r_x,\ c_y = c_x
  \end{equation}
  $x^* \not= 0$时
  \begin{equation}
    r_y = min\{r_x, r_c\}, \quad
    c_y =
    \begin{cases}
      c_x \quad (r_x < r_c) \\
      c_c \quad (r_c < r_x) \\
      max\{c_x, c_c\} \quad (r_x = r_c) \\
    \end{cases}
  \end{equation}
\end{CJK}
\end{document}
