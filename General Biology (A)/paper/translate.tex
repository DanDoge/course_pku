\documentclass{article}
\usepackage[a4paper, total={6in, 9in}]{geometry}
\usepackage{fancyhdr}
\usepackage{amsmath}
\usepackage{algorithm}
\usepackage{algpseudocode}
\usepackage{CJK}
\usepackage{graphicx}

\newtheorem{theorem}{Theorem}
\newtheorem{lemma}{Lemma}
\newtheorem{proof}{Proof}[section]

\pagestyle{fancy}
\fancyhf{}
\rhead{1600017857}
\lhead{Homework(1)}
\rfoot{}
\cfoot{Page \thepage}

\begin{CJK}{UTF8}{gbsn}

\title{论文翻译}
\date{2018-03-10}
\author{
  黄道吉-1600017857
}

\begin{document}

\maketitle

\section{论文信息}
  \paragraph{标题} 哺乳动物呼吸体的结构
  \paragraph{作者} Jinke Gu, Meng Wu, Runyu Guo, Kaige Yan, Jianlin Lei, Ning Gao \& Maojun Yang. 其中前三位为共同一作
  \paragraph{} 清华大学生命科学学院, 结构生物学高精尖创新中心, 清华大学-北京大学生命科学联合中心, 蛋白质科学教育部重点实验室
\section{摘要}
  呼吸链化合物\uppercase\expandafter{\romannumeral1}, \uppercase\expandafter{\romannumeral3} 和 \uppercase\expandafter{\romannumeral4}(CI, CIII, 和 CIV) 存在于细菌膜和线粒体内膜中, 并且具有转移电子, 为化合物V合成ATP建立质子梯度的作用. 呼吸链化合物可以合成超负荷物(SC), 但是它们的精确排列是未知的. 这里我们提供由猪心中纯化出的主要是1.7兆道尔顿的SCI$_1$III$_2$IV$_1$的5.4\AA低温电子显微镜结构. CIII二聚体和CIV结合在L形CI的同一侧, 它们的跨膜结构域基本对齐, 来形成一个跨膜盘. 与自由CI对比, 在呼吸体中的CI因为与CIII和CIV的相互作用而更加紧凑. CI的NDUFA11和NDUF89亚基有助于CI和CIII的寡聚化. 呼吸体的结构提供了关于线粒体中呼吸链化合物精确排列的信息.

\section{正文}
存在于线粒体内膜或嵴中的线粒体呼吸链化合物在能量转换中具有重要作用. 呼吸链化合物包括四种多亚基化合物: 复合物I(NADH), 复合物II(琥珀酸脱氢酶), 复合物III(细胞色素$bc_1$复合物)和复合物IV(细胞色素$c$氧化酶). 三种复合物(CI, CIII 和 CIV)共同为复合物V(CV, ATP合成酶)合成ATP提供跨线粒体内膜的质子梯度
\section{参考文献}
  Jinke Gu, Meng Wu, Runyu Guo, Kaige Yan, Jianlin Lei, Ning Gao \& Maojun Yang. (2016) The architecture of the mammalian respirasome. Nature. 537: 639-643.
\end{CJK}
\end{document}
