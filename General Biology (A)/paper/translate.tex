\documentclass{article}
\usepackage[a4paper, total={6in, 9in}]{geometry}
\usepackage{fancyhdr}
\usepackage{amsmath}
\usepackage{algorithm}
\usepackage{algpseudocode}
\usepackage{CJK}
\usepackage{graphicx}

\newtheorem{theorem}{Theorem}
\newtheorem{lemma}{Lemma}
\newtheorem{proof}{Proof}[section]

\pagestyle{fancy}
\fancyhf{}
\rhead{1600017857}
\lhead{paper translation}
\rfoot{}
\cfoot{Page \thepage}

\begin{CJK}{UTF8}{gbsn}

\title{论文翻译}
\date{2018-03-13}
\author{
  黄道吉-1600017857
}

\begin{document}

\maketitle

\section{论文信息}
  \paragraph{标题} 哺乳动物呼吸体的结构
  \paragraph{作者} Jinke Gu, Meng Wu, Runyu Guo, Kaige Yan, Jianlin Lei, Ning Gao \& Maojun Yang. 其中前三位为共同一作
  \paragraph{} 清华大学生命科学学院, 结构生物学高精尖创新中心, 清华大学-北京大学生命科学联合中心, 蛋白质科学教育部重点实验室
\section{摘要}
  \paragraph{}
    呼吸链化合物\uppercase\expandafter{\romannumeral1}, \uppercase\expandafter{\romannumeral3} 和 \uppercase\expandafter{\romannumeral4}(CI, CIII, 和 CIV) 存在于细菌膜和线粒体内膜中, 并且具有转移电子, 为化合物V合成ATP建立质子梯度的作用. 呼吸链化合物可以合成超负荷物(SC), 但是它们的精确排列是未知的. 这里我们提供由猪心中纯化出的主要是1.7兆道尔顿的SCI$_1$III$_2$IV$_1$的5.4\AA低温电子显微镜结构. CIII二聚体和CIV结合在L形CI的同一侧, 它们的跨膜结构域基本对齐, 来形成一个跨膜盘. 与自由CI对比, 在呼吸体中的CI因为与CIII和CIV的相互作用而更加紧凑. CI的NDUFA11和NDUF89亚基有助于CI和CIII的寡聚化. 呼吸体的结构提供了关于线粒体中呼吸链化合物精确排列的信息.

\section{正文}
  \paragraph{}
    存在于线粒体内膜或嵴中的线粒体呼吸链化合物在能量转换中具有重要作用. 呼吸链化合物包括四种多亚基化合物: 复合物I(NADH), 复合物II(琥珀酸脱氢酶), 复合物III(细胞色素$bc_1$复合物)和复合物IV(细胞色素$c$氧化酶). 三种复合物(CI, CIII 和 CIV)共同为复合物V(CV, ATP合成酶)合成ATP提供跨线粒体内膜的质子梯度. 呼吸链化合物和CV一起形成氧化磷酸化(OXPHOS)系统. 线粒体呼吸链化合物的功能障碍产生活性氧或氮, 这种功能障碍牵连许多人类疾病, 包括阿尔兹海默症, 帕金森症, 多发性硬化症, 弗里德赖希共济失调和肌萎缩性侧索硬化症.
  \paragraph{}
    除了形成随机分散在膜中的单独的复合物之外, 许多证据表明这些复合物也可以组成特定的活性超结构. 超复合物的组成变化多端. 不同物种中都有报道不同种类数目形式组成的超复合物的存在. 从细菌中纯化出的CIII和CIV长期以来被忽视, 直到线粒体研究领域出现一种称作蓝色原生PAGE(BNGA)的策略. 在BNPA分析毛地黄皂苷增容制剂过程中发现的高分子条带被发现含有单个复合物组成的稳定组合. 在没有多亚基CI的酿酒酵母(\textit{Saccharomyces cerevisiae})中, 可以提取含有III$_{2}$IV$_{1}$(750kDa)和III$_{2}$IV$_{2}$(1000kDa)的稳定复合物. 在拟南芥(\textit{Arabidopsis thaliana})中, 超复合物的主要形式包括I$_{2}$和III$_{2}$. 在哺乳动物中, 可以从牛心中获取SCI$_{1}$III$_{2}$IV$_{1}$, 并且CII还认为和小鼠肝线粒体超化合物合成有关.
  \paragraph{}
    在牛心脏中, 毛地黄皂苷溶解后, ~80\%的CI分散到超复合物中, 而只有~20\%的CI以游离形式存在. 大约三分之二的CIII被发现在各种形式的超复合物中, 而CIV主要(~85\%)以游离形式存在. 这些数据表明呼吸链复合物可能在线粒体中有结构的互相依赖性. 的确, 以往的研究表明在哺乳动物细胞中组装CIII和CIV需要稳定的CI.
  \paragraph{}
    有几个研究组认为, 在呼吸链中, 包含CI, CIII, CIV的超复合物会组成呼吸串: 在呼吸串中不同的超复合物在线粒体内膜嵴组织成有序的线性图案, 尽管我们对这个组织的了解十分有限. 在这些超分子组装状态中, SCI$_{1}$III$_{2}$IV$_{1}$的结构引发很大的兴趣, 也是研究最深入的. 然而, 过去曾缺乏复杂的相互作用的精确解释, 主要囿于以前结构的有限的分辨率. 在这项研究中, 我们从猪心中纯化出SCI$_{1}$III$_{2}$IV$_{1}$的主要形式, 并使用单粒子冷冻电子显微镜(cryo-EM)获得了5.4\AA分辨率的结构. 我们的结构揭示了SCI$_{1}$III$_{2}$IV$_{1}$中复合物的精确位置, 并建立出呼吸链复合物之间的复杂相互作用.

\section{蛋白质纯化与结构测定}
  \paragraph{}
    基本如同之前所述, 超化合物从牛, 猪, 大鼠和小鼠心脏纯化出来. 从猪心中提取出来的用毛地黄皂苷溶解的超化合物用作最终的样品制备和机构测定. BNPA分析检测到几种超复合物, 其中主要的一种形成1.7MDa的迁移, 这和对SCI$_{1}$III$_{2}$IV$_{1}$的大小估计一致. 这条SCI$_{1}$III$_{2}$IV$_{1}$可被活性CI的特异染料硝基蓝四唑(NBT)染色, 表明这种超复合物中的CI具有NADH脱氢酶的活性.
  \paragraph{}
    1.7MDa呼吸体的初始三维参考建模由负染色电子显微镜(EM)创建. 正如扩展参考数据图2中所示, 初始模型有类似于以往描述的F型结构. 高分辨率的图像由配备了Falcno II的Titan Krios TEM记录. 原始冷冻电镜的分类使得2D分类平均分辨率较好, 一些耳机结构特征清晰可辨. 在对颗粒进行三维分类后, 一部分粒子得以高分辨率细化, 产生总体分辨率5.4\AA的3D密度图, 图中部分辨率更高, 外围分辨率较低.
  \paragraph{}
    F型密度图可以清晰的分为三部分, 一个细长部分位于两个球形部分的下面. 结构的俯视图和正视图的直径分别为300\AA和190\AA. 通过使用CI和CIII的两种不同掩码的子区域精修进一步将CI和CIII的密度图分别改进到~3.9-4.0\AA. 从密度图中可以容易识别跨膜区, 因为它有许多密度棒. 局部解析图显示CIV以外的跨膜螺旋是分辨率最高的区域之一. 随后将牛CI的冷冻电镜5\AA模型和CIII同型二聚体和CIV的单体的高分辨率晶体结构导入图像中得到很高的相关系数. 使用Coot进一步手动优化得到的模型.

\section{CI的分配}
  \paragraph{}
    CI的结果和功能被深入研究了数十年. 哺乳动物的CI是线粒体内膜最大, 最复杂的酶. CI的整体高度约为190\AA, 长度约为250\AA, 包含44个不同的亚基核两份NDUFAB1. 报道称14个保守的(什么叫保守的(conserved)?)催化能量转换反应的亚基排成L型结构. 7个基质核心亚基被10个额外亚基包围, 突出到膜外形成基质臂. 7个膜臂核心亚基包在线粒体内膜内, 另有21个额外亚基包围它们. 14个额外结构域至少有一个跨膜结构域. CI的3.97\AA密度图使我们能够用核心亚基的大部分残基的侧链建立模型, 尤其是在膜臂区域. 侧链的密度在紧密堆积的亚单元那里解决的很好, 但周边区域的边缘密度不那么有序. 在这个结构中, 我们能够准确定位14个核心亚基和20个额外亚基的位置. 此外, 我们还建立了另外17个骨干模型来放到没有占用的密度中, 这些密度应该属于另外11个未分配的额外成分和/或指定蛋白的未分配区域. 总的来说, CI的结构在膜结构域中包含77个跨膜螺旋. 与先前报道的牛结构相比, 我们的结构中没有检测到ND5旁边的膜臂末端的一个跨膜螺旋.
  \paragraph{}
    CI可以分为4个功能模块. 基质臂远端形成N模块, 它包括核心亚单位NDUFV1, NDUFV2和NDUFS1, 以及额外亚基NDUFS6和NDUFA2. N模块含有氧化NADH的FMN分子. 基质臂的近端一半形成另一个功能模块, 它对接在由核心亚基NDUFS2, NDUFS3, NDUFS7和NDUFS8以及额外亚基NDUFA5, NDUFA6, NDUFA9和NDUFAB1组成的膜臂上. NDUFS4额外亚基和两种基质模块的所有核心亚基通过桥接在一起相互作用. 膜臂包括另外两个模块. 核心亚基ND5和ND4连同若干额外成分构成了两个远端逆向运输质子泵(two distal antiporter-like proton pump)模块($P_p$), 它包含两个额外的质子泵并于基质臂结合.

\section{CIII和CIV的分配}
  \paragraph{}
    除了CI的膜臂之外, 具有近似双重对称性的密度区域(高160\AA, 宽140\AA)可以容易的分配给CIII同源二聚体. 这种分配与二聚CIII是线粒体呼吸链的核心部件, 从ubihydroquinone转移电子到cytochrome $c$生成跨线粒体内膜的质子梯度是一致的. 之前的研究表明, CIII中的Rieske铁硫蛋白亚基的功能域在不同的结构形式中有不同的位置, 称为'$c$', '$b$', 和'$int$'(中间)状态, 对应于从[2Fe-2S]簇到haem $c_1$或$b_L$的不同距离. 在最终精绘密度图中, [2Fe-2S]到haem $c_1$和haem $b_L$的相对距离分别为30\AA和27\AA
  \paragraph{}
    在这个结构中, 每一个CIII单体包含11种不同多肽, 总重大约240kDa. CIII同型二聚体的两个轴向垂直于CI膜臂的膜平面. CIII二聚体的基质结构域在基质一面突出大约75\AA, 直面CI的亲水臂. 在膜间隙处, CIII 外表面几乎和同侧CI膜臂中额外蛋白质的外表面水平. 然而, 在膜间隙CIII的外表面上预期的细胞色素$c$的结合点位处没有密度, 推测细胞色素$c$可能在样品纯化过程中丢失.
  \paragraph{}
    位于CI的膜臂末端, 与CIII二聚体相邻的矩形密度大致有90, 60, 120\AA那么大, 和CIV单体的尺寸一致. CIV晶体结构的13个亚基的刚体拟合表明, COX4I1和COX5A亚基向基质侧突出, 预期的细胞色素$c$的结合位点位于膜间一侧. CIV在晶体结构凹侧一面的二聚界面面向整个超复合物的外侧, 提高了呼吸体中的CIV可能在高阶组装中形成同型二聚体的可能性.
  \paragraph{}
    据报道, SCI$_{1}$III$_{2}$IV$_{1}$包含69种不同的多肽, 它们由来自CI的45种, CIII二聚体的22种和CIV的14种, 合计81种蛋白链构成. 我们的密度图表明仍存在未分配的额外成分和/或CI的未分配区域, 我们的中等分辨率的密度图不允许我们确定它们的成分. 即便如此, 我们可以在CI的跨膜区域和周边未占用的密度中构建17条主链. 这使得我们的呼吸体的结构中有131个跨膜螺旋, 其中77个来自CI, 26个来自CIII, 28个来自CIV. 三个复合物的跨膜区域形成一个巨大的跨膜盘, 它在膜间有一个平坦的平面.

\section{CI和CIII的相互作用}

\section{参考文献}
  Jinke Gu, Meng Wu, Runyu Guo, Kaige Yan, Jianlin Lei, Ning Gao \& Maojun Yang. (2016) The architecture of the mammalian respirasome. Nature. 537: 639-643.
\end{CJK}
\end{document}
