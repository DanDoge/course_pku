\documentclass{article}
\usepackage[a4paper, total={6in, 9in}]{geometry}
\usepackage{fancyhdr}
\usepackage{amsmath}
\usepackage{algorithm}
\usepackage{algpseudocode}
\usepackage{CJK}
\usepackage{graphicx}

\newtheorem{theorem}{Theorem}
\newtheorem{lemma}{Lemma}
\newtheorem{proof}{Proof}[section]

\pagestyle{fancy}
\fancyhf{}
\rhead{1600017857}
\lhead{Homework(1)}
\rfoot{}
\cfoot{Page \thepage}

\begin{CJK}{UTF8}{gbsn}

\title{论文翻译}
\date{2018-03-10}
\author{
  黄道吉-1600017857
}

\begin{document}

\maketitle

\section{论文信息}
  \paragraph{标题} 哺乳动物呼吸体的结构
  \paragraph{作者} Jinke Gu, Meng Wu, Runyu Guo, Kaige Yan, Jianlin Lei, Ning Gao \& Maojun Yang. 其中前三位为共同一作
  \paragraph{} 清华大学生命科学学院, 结构生物学高精尖创新中心, 清华大学-北京大学生命科学联合中心, 蛋白质科学教育部重点实验室
\section{摘要}
  \paragraph{}
    呼吸链化合物\uppercase\expandafter{\romannumeral1}, \uppercase\expandafter{\romannumeral3} 和 \uppercase\expandafter{\romannumeral4}(CI, CIII, 和 CIV) 存在于细菌膜和线粒体内膜中, 并且具有转移电子, 为化合物V合成ATP建立质子梯度的作用. 呼吸链化合物可以合成超负荷物(SC), 但是它们的精确排列是未知的. 这里我们提供由猪心中纯化出的主要是1.7兆道尔顿的SCI$_1$III$_2$IV$_1$的5.4\AA低温电子显微镜结构. CIII二聚体和CIV结合在L形CI的同一侧, 它们的跨膜结构域基本对齐, 来形成一个跨膜盘. 与自由CI对比, 在呼吸体中的CI因为与CIII和CIV的相互作用而更加紧凑. CI的NDUFA11和NDUF89亚基有助于CI和CIII的寡聚化. 呼吸体的结构提供了关于线粒体中呼吸链化合物精确排列的信息.

\section{正文}
  \paragraph{}
    存在于线粒体内膜或嵴中的线粒体呼吸链化合物在能量转换中具有重要作用. 呼吸链化合物包括四种多亚基化合物: 复合物I(NADH), 复合物II(琥珀酸脱氢酶), 复合物III(细胞色素$bc_1$复合物)和复合物IV(细胞色素$c$氧化酶). 三种复合物(CI, CIII 和 CIV)共同为复合物V(CV, ATP合成酶)合成ATP提供跨线粒体内膜的质子梯度. 呼吸链化合物和CV一起形成氧化磷酸化(OXPHOS)系统. 线粒体呼吸链化合物的功能障碍产生活性氧或氮, 这种功能障碍牵连许多人类疾病, 包括阿尔兹海默症, 帕金森症, 多发性硬化症, 弗里德赖希共济失调和肌萎缩性侧索硬化症.
  \paragraph{}
    除了形成随机分散在膜中的单独的复合物之外, 许多证据表明这些复合物也可以组成特定的活性超结构. 超复合物的组成变化多端. 不同物种中都有报道不同种类数目形式组成的超复合物的存在. 从细菌中纯化出的CIII和CIV长期以来被忽视, 直到线粒体研究领域出现一种称作蓝色原生PAGE(BNGA)的策略. 在BNPA分析毛地黄皂苷增容制剂过程中发现的高分子条带被发现含有单个复合物组成的稳定组合. 在没有多亚基CI的酿酒酵母(\textit{Saccharomyces cerevisiae})中, 可以提取含有III$_{2}$IV$_{1}$(750kDa)和III$_{2}$IV$_{2}$(1000kDa)的稳定复合物. 在拟南芥(\textit{Arabidopsis thaliana})中, 超复合物的主要形式包括I$_{2}$和III$_{2}$. 在哺乳动物中, 可以从牛心中获取SCI$_{1}$III$_{2}$IV$_{1}$, 并且CII还认为和小鼠肝线粒体超化合物合成有关.
  \paragraph{}
    在牛心脏中, 毛地黄皂苷溶解后, ~80\%的CI分散到超复合物中, 而只有~20\%的CI以游离形式存在. 大约三分之二的CIII被发现在各种形式的超复合物中, 而CIV主要(~85\%)以游离形式存在. 这些数据表明呼吸链复合物可能在线粒体中有结构的互相依赖性. 的确, 以往的研究表明在哺乳动物细胞中组装CIII和CIV需要稳定的CI.
  \paragraph{}
    有几个研究组认为, 在呼吸链中, 包含CI, CIII, CIV的超复合物会组成呼吸串: 在呼吸串中不同的超复合物在线粒体内膜嵴组织成有序的线性图案, 尽管我们对这个组织的了解十分有限. 在这些超分子组装状态中, SCI$_{1}$III$_{2}$IV$_{1}$的结构引发很大的兴趣, 也是研究最深入的. 然而, 过去曾缺乏复杂的相互作用的精确解释, 主要囿于以前结构的有限的分辨率. 在这项研究中, 我们从猪心中纯化出SCI$_{1}$III$_{2}$IV$_{1}$的主要形式, 并使用单粒子冷冻电子显微镜(cryo-EM)获得了5.4\AA分辨率的结构. 我们的结构揭示了SCI$_{1}$III$_{2}$IV$_{1}$中复合物的精确位置, 并建立出呼吸链复合物之间的复杂相互作用.

\section{蛋白质纯化与结构测定}
  


\section{参考文献}
  Jinke Gu, Meng Wu, Runyu Guo, Kaige Yan, Jianlin Lei, Ning Gao \& Maojun Yang. (2016) The architecture of the mammalian respirasome. Nature. 537: 639-643.
\end{CJK}
\end{document}
