\documentclass{article}
\usepackage[a4paper, total={6in, 9in}]{geometry}
\usepackage{fancyhdr}
\usepackage{amsmath}
\usepackage{amsthm}
\usepackage{algorithm}
\usepackage{algpseudocode}
\usepackage{CJK}
\usepackage{graphicx}

\pagestyle{fancy}
\fancyhf{}
\rhead{1600017857}
\lhead{Exam-1}
\rfoot{}
\cfoot{Page \thepage}

\title{Exam-1}
\date{2018-03-02}
\author{
  \begin{CJK}{UTF8}{gbsn}
    黄道吉-1600017857
  \end{CJK}
}

\begin{document}
\begin{CJK}{UTF8}{gbsn}
\section{第十二题}
  使用类似辗转相除法的方法:
  \begin{equation*}
    \begin{aligned}
      r_0 &= m \mod n \\
      r_1 &= m - r_0s_0\quad  s.t.\quad 0 \leq r_1 < r_0 \\
      \dots & \\
      r_k &= m - r_{k - 1}s_{k - 1} \\
    \end{aligned}
  \end{equation*}
  直到某一个$r_{k + 1}$为零为止, 则类似辗转相除法正确性的证明, 可得($m$, $n$)也是上述$r_i$的因数, 且因为$(m, n) \leq r_k < r_{k - 1} < \dots < r_0$, 则必存在一个$k$, $s.t.\ r_k = (m, n)$(由良序性保证).
  \newline
  下证明$(2^m - 1, 2^n + 1) = (2^m - 1, 2^{r_0} + 1)$(不妨设$n > m$, $n \leq m$时自然成立):
  \begin{equation*}
    \begin{aligned}
      (2^m - 1, 2^n + 1) &= (2^m - 1, 2^n + 1 - 2^{n - m}(2^m - 1)) \\
      &= (2^m - 1, 2^{n - m} + 1) \\
      &= \dots \\
      &= (2^m - 1, 2^{n \mod m} + 1) \\
    \end{aligned}
  \end{equation*}
  \newline
  下证明$(2^m - 1, 2^{r_i} + 1) = (2^m - 1, 2^{r_{i + 1}} + 1)$:
  \begin{equation*}
    \begin{aligned}
      (2^m - 1, 2^{r_i} + 1) &= (2^m - 1, 2^{m - r_i}(2^{r_i} + 1)) \quad (because (2^m - 1, 2^{m - r_i}) = 1) \\
      &= (2^m - 1, 2^{m - r_i}(2^{r_i} + 1) - (2^m - 1)) \\
      &= (2^m - 1, 2^{m - r_i} + 1) \\
      &= \dots \\
      &= (2^m - 1, 2^{r_{i + 1}} + 1) \\
    \end{aligned}
  \end{equation*}
  则只需证明, $(2^{m_1d} - 1, 2^d + 1) = 1, $其中$d = (m, n)\ m = m_1d$, 易得$m_1, d$都是奇数(因为$m$是奇数):
  \newline
  由分解式$2^{m_1d} - 1 = (2^d - 1)(2^{(m_1 - 1)d} + ... + 2^d + 1)$, 且:
  \begin{equation*}
    \begin{aligned}
      (2^d + 1, 2^{(m_1 - 1)d} + ... + 2^d + 1) &= (2^d + 1, 2^{(m_1 - 3)d} + ... + 2^d + 1) \\
      &= \dots \\
      &= (2^d + 1, 2^{2d} + 2^d + 1) \\
      &= (2^d + 1, 1) \\
      &= 1 \\
    \end{aligned}
  \end{equation*}
  , 并且$(2^d + 1, 2^d - 1) = (2^d + 1, 2) = (1, 2) = 1$, 所以有:
  \begin{equation*}
    (2^d + 1, (2^d - 1)(2^{(m_1 - 1)d} + ... + 2^d + 1)) = 1
  \end{equation*}
  证毕.
  $\hfill{} \qed$
\end{CJK}
\end{document}
