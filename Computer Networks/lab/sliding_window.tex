\documentclass{article}
\usepackage[UTF8]{ctex}
\usepackage{geometry}
\usepackage{float}
\usepackage{fancyhdr}
\usepackage{extramarks}
\usepackage{amsmath}
\usepackage{amsthm}
\usepackage{amsfonts}
\usepackage{tikz}
\usepackage{multicol}
\usepackage[ruled,vlined]{algorithm2e}
\usepackage{algpseudocode}
\usetikzlibrary{automata,positioning}

\geometry{left = 2cm, right = 2cm, top = 2cm, bottom = 2cm}

\author{黄道吉}
\title{滑动窗口协议实验报告}
\date{\today}

\begin{document}

\maketitle

滑动窗口协议是链路层数据传输的重要协议。这次实验要求在模拟的数据链路层环境中,实现下列三个协议中发送端的功能
\begin{itemize}
    \item 停等协议
    \item 回退N滑动窗口协议
    \item 选择重传协议
\end{itemize}

\section{数据结构}

除了提供的帧和帧头之外,我们另外维护两个数据结构分别用来存储 1)未发送的数据 和 2)已经发送(可能需要重发的数据)。其中未发送的数据只需要能够保证先进先出,用queue实现,已发送的数据在全部重发时需要一一迭代访问,最好实现成deque。两个数据结构的用法在协议实现部分有详细的说明。

\section{协议实现}

\subsection{停等协议}

按照停等协议的要求,每次只能发送一个帧,并保存刚刚发送的帧。因此
\begin{itemize}
    \item 当遇到超时信号时,重传刚刚发送的帧
    \item 当遇到发送信号时,将需要发送的帧入队。如果已发送的队列为空,即是可以发送,则从待发送队列头取一帧发送。
    \item 当收到接受信号时,清空已发送队列。如果待发送队列不空,则可以再发送一帧。
\end{itemize}

\subsection{回退N滑动窗口协议}

按照回退N滑动窗口协议的要求,维护一个最大长度为N的已发送队列。因此
\begin{itemize}
    \item 当遇到超时信号时,重传所有已发送的帧
    \item 当遇到发送信号时,将需要发送的帧入队。如果已发送的队列长度小于N,即是可以发送,则从待发送队列头取一帧发送,直到已发送N帧。
    \item 当收到接受信号时,清空序号不等于接受帧序号的帧。如果待发送队列不空,则可以再发送帧,直到已发送N帧。
\end{itemize}

\subsection{选择重传协议}

按照选择重传协议的要求,维护一个最大长度为N的已发送队列。因此
\begin{itemize}
    \item 当遇到超时信号时,重传需要重传的帧
    \item 当遇到发送信号时,将需要发送的帧入队。如果已发送的队列长度小于N,即是可以发送,则从待发送队列头取一帧发送,直到已发送N帧。
    \item 当收到接受信号时,如果是成功接收,则清空序号不等于接受帧序号的帧。如果待发送队列不空,则可以再发送帧,直到已发送N帧;如果是未成功接受,按照上课说的解决方法,重传所有帧
\end{itemize}

\end{document}
