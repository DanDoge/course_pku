\documentclass{article}
\usepackage[UTF8]{ctex}
\usepackage{geometry}

\usepackage{listings}
\usepackage{float}
\usepackage{fancyhdr}
\usepackage{extramarks}
\usepackage{amsmath}
\usepackage{amsthm}
\usepackage{amsfonts}
\usepackage{tikz}
\usepackage{multicol}
\usepackage[ruled,vlined]{algorithm2e}
\usepackage{algpseudocode}
\usetikzlibrary{automata,positioning}

\geometry{left = 2cm, right = 2cm, top = 2cm, bottom = 2cm}

\author{黄道吉}
\title{ipv4协议转发实验报告}
\date{\today}

\begin{document}

\maketitle

IPv4协议是互联网的核心协议。这次实验要求实现IPv4协议中转发数据包的部分。与上次实验不同的是,本次实验实现的主要是路由器的功能而不是主机的功能。实验要求在原有收发功能之外,另加IPv4分组的转发代码,要求能够实现路由表的数据结构,实现路由器选择路由的功能。

\section{数据结构}

这次实验的中心是设计路由表的结构。在我的实现中,路由表被设置为一个路由消息(route\_msg)到下一跳的映射,利用c++的标准库实现。另外增加一个函数判断某个地址是否落在路由消息的目的地址之内。这样对路由表的初始化、添加和查找方法分别实现为:
\begin{itemize}
    \item 初始化:设置路由表为空。
    \item 添加路由消息:先查找路由表中有没有冗余表项,没有的话再添加进路由表中。注意对于masklen不相同的路由信息,即便目的地址相同,也不能够认为他们相互有包含/被包含的关系,因为在进行最大长度匹配时会忽略掉masklen更小的表项。为尽量压缩路由表的大小,实现的时候认为目的地址相同(加掩码的意义下),并且 a) 掩码长度相同 或者 b) 待添加的表项掩码更长但是下一条相同的路由信息相同 的路由信息是冗余的信息。在a)情况下,它的目的地址和某个已经存在的表项完全相同,b)的情况下它的目的地址和下一跳被已有表项完全覆盖。
    \item 查找匹配的路由:迭代map中各个元素,查看目的地址是否和路由消息的目的地址相等(在加上掩码的意义下),在所有相等的路由消息中选择最长匹配的。
\end{itemize}

\section{实现细节}

本节介绍我的实现的细节和只参考手册和查阅资料没有解决的问题。

\subsection{函数逻辑}

这部分介绍主函数stud\_fwd\_deal实现的功能。在收到需要转发/接受的分组时,首先判断分组的存活时间,为0的应当抛弃。因为手册中只给出两种丢包的错误,所以不再判断分组头部的其他域。对于目的地址是自己(本机ip地址或广播地址)的数据包应当接收。对于其他的数据分组,则需要迭代路由表每一个表项,找到最大长度匹配的表项(因此每一次都需要迭代整个路由表),决定转发的接口。最后参考上一个实验的代码,重组分组头部(ttl减一,计算checksum),送到下层。

\subsection{残存问题}

实现的过程中仍然没有解决的问题有
\begin{itemize}
    \item 和上一次实验相同,本机地址是否需要包含诸如127.0.0.0/8的地址,还是测试样例中并不会包含这些地址
    \item 路由消息的各个域的字节序是什么
\end{itemize}

此外,为了更加有效的压缩路由表的体积,一个可行的方法可能是添加表项时检查/定期扫描有没有可以合并的表项,e.g. <162.0.0.0/8, 0> 和 <163.0.0.0/8, 0> 可以合并为 <162.0.0.0/7, 0>,但并不确定 a) 这种合并方式在测试样例下的正确性,e.g. 合并后再添加表项<162.0.0.0/8, 1>时并不会认为和已有的表项冲突,和 b) 计算的复杂性是否合适。



\section{代码实现}


\lstset{language=C++,
%backgroundcolor=\color{write},
basicstyle=\footnotesize,
keywordstyle=\color{blue}\bfseries,
commentstyle=\color{gray},
}

\lstset{breaklines}
\lstset{extendedchars=false}

\begin{lstlisting}
#include <map>
using namespace std; // for its map data structure

/* API provided by system */

void fwd_LocalRcv(char *pBuffer, int length);
void fwd_SendtoLower(char *pBuffer, int length, unsigned int nexthop);
void fwd_DiscardPkt(char * pBuffer, int type);
UINT32 getIpv4Address( );

/* routetable: route_message -> nexthop */
map<stud_route_msg, int> routetable;

/* helper function(s) */
int helper_route_include(unsigned int destination_address, stud_route_msg route_message){
    // destination_address == route_message.dest under a certain mask
    return (destination_address - route_message.dest) & ((~0) << (route_message.masklen)) == 0;
}

int stud_fwd_deal(char * pBuffer, int length){

/* code to check ttl and dstaddr is borrowed from ipv4_receive.c from here */

    unsigned int time_to_live = ((unsigned int)pBuffer[8]);
    if(time_to_live == 0){
        ip_DiscardPkt(pBuffer, STUD_FORWARD_TEST_TTLERROR);
        return -1;
    }

    unsigned int destination_address = ntohl(*(unsigned int* )(pBuffer + 16)); // might be wrong

/* to here */

    if(destination_address == getIpv4Address() || destination_address != (~0)){
        fwd_LocalRcv(pBuffer, length);
        return 0;
    }


    int nexthop = -1, maxlengthmatch = -1; // assume nexthop != -1 for all route_message
    for(map<stud_route_msg, int>::iterator it = routetable.begin();
        it != routetable.end();
        it++){
        /* assume masklen is the 'n' in x.y.z.m/n */
        if(helper_route_include(destination_address, *it)){
            // a match found!
            if(maxlengthmatch < 32 - it->masklen){
                maxlengthmatch = 32 - masklen;
                nexthop = it-> nexthop;
            }
        }
    }

    if(nexthop == -1){
        /* no match found */
        fwd_DiscardPkt(pBuffer, STUD_FORWARD_TEST_NOROUTE);
        return -1;
    }


    unsigned char* total_buffer = malloc(length);
    memcpy(total_buffer, pBuffer, length);
    total_buffer[8] = time_to_live - 1;

/* code to compute checksum is borrowed from ipv4_receive.c from here */
    unsigned int header_checksum = 0;
    for(int i = 0; i < 20; i += 2){
        header_checksum += (total_buffer[i] << 8) + total_buffer[i + 1];
    }
    while(header_checksum >> 16){
        header_checksum == (header_checksum >> 16) + (header_checksum & 0xFFFF);
    }
    header_checksum = ~header_checksum;
    total_buffer[10] = header_checksum >> 8;
    total_buffer[11] = header_checksum & 0xFF;
/* to here */
    fwd_SendtoLower(total_buffer, length, nexthop);

    return 0;
}

void stud_route_add(stud_route_msg* proute){
    /* subject to change: what is the byteorder of dest, masklen and nexthop? */
    for(map<stud_route_msg, int>::iterator it = routetable.begin();
        it != routetable.end();
        it++){
        /* assume masklen is the 'n' in x.y.z.m/n */
        if(helper_route_include(proute->destination_address, *it)
           && (proute->masklen == it->masklen
               || (proute->masklen >= it->masklen && proute->nexthop == it->nexthop)
              )
           ){
            // a match found! no need to update route table
            return ;
        }
    }
    routetable.insert(std::pair<stud_route_msg, int>(*proute, proute->nexthop));
    return ;
}

void stud_Route_Init(){
    routetable.clear();
    return ;
}

\end{lstlisting}


\end{document}
